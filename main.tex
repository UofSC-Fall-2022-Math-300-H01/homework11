\documentclass[12pt]{amsart}
\usepackage{amsmath}
\usepackage{amsthm}
\usepackage{amsfonts}
\usepackage{amssymb}
\usepackage{ebproof}
\usepackage[margin=1in]{geometry}
\usepackage{hyperref}
\hypersetup{
    colorlinks=true,
    linkcolor=blue
}

\theoremstyle{definition}
\newtheorem{theorem}{Theorem}[section]
\newtheorem*{theorem*}{Theorem}
\newtheorem{lemma}[theorem]{Lemma}
\newtheorem{definition}[theorem]{Definition}
\newtheorem{corollary}[theorem]{Corollary}
\newtheorem{proposition}[theorem]{Proposition}
\newtheorem{conjecture}[theorem]{Conjecture}
\newtheorem{remark}[theorem]{Remark}
\newtheorem{example}[theorem]{Example}
\newtheorem{problem}[theorem]{Problem}
\newtheorem{notation}[theorem]{Notation}
\newtheorem{question}[theorem]{Question}
\newtheorem{caution}[theorem]{Caution}

\begin{document}

\title{Homework}

\maketitle

\begin{enumerate}
	\item Translate the statement and proof of problem1 in Hw10.lean to 
		a pen-and-paper statement and proof. 

	\item Translate the statement and proof of problem2 in Hw10.lean to 
		a pen-and-paper statement and proof. 

	\item Translate the statement and proof of problem3 in Hw10.lean to 
		a pen-and-paper statement and proof. 

	\item Below is a theorem and proof. Translate the statement as a 
		theorem problem4 and give a proof of it in Lean in Hw10.lean. 

	\begin{theorem*}
		Let $f : A \to B$ and $g : B \to C$ be 
		functions. If $d : B \to A$ is a left inverse to $f$ and  
		$e : C \to B$ is a left inverse to $g$, then $d \circ e$ is a left 
		inverse to $g \circ f$. 
	\end{theorem*}
		
	\begin{proof}
		To be a left inverse we need to show that 
		\begin{displaymath}
			(d \circ e) \circ (g \circ f) = \operatorname{id}_A 
		\end{displaymath}
		for all $a$. Using the fact that composition is associative, 
		we can rewrite the left-hand side as 
		\begin{displaymath}
			(d \circ e) \circ (g \circ f)  = d \circ (e \circ g) \circ f
		\end{displaymath}
		By definition of a left inverse, we know that $e \circ g = 
		\operatorname{id}_B$ so we can rewrite again 
		\begin{displaymath}
			d \circ (e \circ g) \circ f = d \circ \operatorname{id}_B 
			\circ f 
		\end{displaymath}
		Since composition with the identity function is the identity we 
		have 
		\begin{displaymath}
			d \circ \operatorname{id}_B \circ f = d \circ f 
		\end{displaymath}
		Finally, as $d$ is a left inverse to $f$, we have 
		\begin{displaymath}
			d \circ f = \operatorname{id}_A
		\end{displaymath}
	\end{proof}
		
	\item Below is a theorem and proof. Translate the statement as a 
		theorem labeled problem5 give a proof of it in Lean in Hw10.lean.

	\begin{theorem*}
		Being in bijection is symmetric. More precisely, if there exists a bijection 
		$A \cong B$, then there also exists a bijection $B \cong A$. 
	\end{theorem*}
	
	\begin{proof}
		Let $f : A \to B$ be a bijection. Since $f$ is a bijection, it has 
		an inverse $f^{-1} : B \to A$ which satisfies $f \circ f^{-1} = 
		\operatorname{id}_B$ and $f^{-1} \circ f = \operatorname{id}_A$. We 
		claim that $f^{-1}$ is also bijection. But the conditions we just 
		listed imply that $f$ is the inverse of $f^{-1}$. Since $f^{-1}$ 
		has an inverse, it is a bijection also. 
	\end{proof}
		
\end{enumerate}
\end{document}
